\chapter{Wprowadzenie}
\label{cha:wprowadzenie}

\section{Cel}

W ramach pracy inżynierskiej powstał dwukołowy robot balansujący. Weryfikacja przyjętych wtedy założeń projektowych uwidoczniła wiele potencjalnych udoskonaleń.

\section{Zakres pracy}

Celem pracy jest budowa robota dwukołowego oraz implemen-
tacja algorytmu predykcyjnego stabilizującego robota w gór-
nym, niestabilnym punkcie równowagi. Konstrukcja mechaniczna będzie bazować na robocie wykonanym w ramach pracy inżynierskiej, zostanei ona przebudowana i dopracowana.

Algorytm będzie rozwiązywał w każdym kroku, odpowiednio sformułowane zadanie
programowania kwadratowego. Jednym z podstawowych elementów pracy będzie też modelowanie oraz identyfikacja modelu matematycznego robota. Algorytm będzie zaimplementowany na platformie STM32F3 Discovery. Zmiana mikrokontrolera wymaga dostosowanie sterownika do pracy w nowych warunkach i napisanie kodu źródłowego. 





\section{Zawartość pracy}
\label{sec:zawartoscPracy}

W rodziale~\ref{cha:pierwszyDokument} przedstawiono....


















