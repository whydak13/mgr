\chapter{Konstrukcja mechaniczna}
\label{cha:KonstrukcjaMechaniczna}

W rozdziale zawarto opis konstrukcji ramy pojazdu oraz jednostki napędowej. Założenia projektowe części mechanicznej to uzyskanie maksymalnej trwałości przy utrzymaniu niskich kosztów pojazdu oraz możliwość łatwej modyfikacji i napraw. 

\section{Rama i rozlokowanie podzespołów}
Podstawowym założeniem podczas projektu ramy robota oraz rozlokowywaniu komponentów było zabezpieczenie przed uszkodzeniami wynikającymi z ewentualnych upadków. Rama została wykonana z płyty pleksiglasowej o grubości 4mm. Pozwoliło to utrzymać niską masę przy stosunkowo dużej wytrzymałości konstrukcji. Poszczególne części sterownika zostały przykręcone do ramy w górnej części pojazdu i zabezpieczone przed uderzeniami dodatkowymi fragmentami ramy.

Zastosowanie dwóch kul podporowych, po obu stronach robota, pozwala także na jazdę w pozycji poziomej. Akumulatory zostały umieszczone w dolnej części pojazdu, rozwiązanie to znacznie obniża środek ciężkości. Pozwala to na bardziej dynamiczną jazdę lecz utrudnia stabilizację w górnym punkcie równowagi. W kolejnych podpunktach omówione zostaną kluczowe elementy mechaniczne konstrukcji. Zdjęcie poglądowe robota zostało przedstawione na rysunku \ref{zdj_robot}

\begin{figure}[h]
	\centering
%	\includegraphics[scale=0.6]{}
	\caption{ Zdjęcie poglądowe robota. Zdjęcie własne}
	\label{zdj_robot}
\end{figure}

\section{Silniki}

Jako element napędowy robota zastosowano dwa silniki krokowe 42HS40-0504.Stosunkowo duży moment obrotowy przy niskiej prędkości obrotowej, możliwość kontroli pozycji w pętli otwartej oraz możliwość częstej zmiany kierunku czynią silniki krokowe zasadnym wyborem do pojazdu tego typu. Zastosowane w poprzedniej iteracji projektu silniki zostały jednak wymienione silniki hybrydowe o większym momencie obrotowym oraz wyższej liczbie kroków na obrót. Podstawowe parametry wykorzystanych silników przedstawione zostały w tabeli \ref{tab:skutecznosc}

\begin{table}
	\centering
	
	\begin{tabular}{|c|r|r|}
	
		\hline
		Model& 42HS40-0504 \\		
		\hline
		Liczba kroków&  200\\      
		\hline                                      
		Napięcie znamionowe &12V   \\       
		\hline                                      
		Pobór prądu na cewkę &  0,5A\\ 
		\hline                                                                         
		Rezystancja cewki & 24Ohm\\
		\hline      
		Moment trzymający& 0,44Nm\\
		\hline      
		Wymiary bez wału& 42 x 42 x 40mm\\
		\hline      
		
		\hline        
	\end{tabular}
	\caption{Parametry silników}
	\label{tab:skutecznosc}	
\end{table}


